\documentclass{article}
\usepackage[utf8]{inputenc}
\usepackage[spanish]{babel}
\usepackage{listings}
\usepackage{graphicx}
\graphicspath{ {images/} }
\usepackage{cite}

\begin{document}

\begin{titlepage}
    \begin{center}
        \vspace*{1cm}
            
        \Huge
        \textbf{Informe 1 Informatica II}
            
        \vspace{0.5cm}
        \LARGE
        Algoritmo de la hoja y 2 tarjetas
            
        \vspace{1.5cm}
            
        \textbf{Andrés Felipe Rendón Villada}
            
        \vfill
            
        \vspace{0.8cm}
            
        \Large
        Despartamento de Ingeniería Electrónica y Telecomunicaciones\\
        Universidad de Antioquia\\
        Medellín\\
        Marzo de 2021
            
    \end{center}
\end{titlepage}

\tableofcontents
\newpage
\section{Sección introductoria}\label{intro}
La elaboración de los algoritmos puede tornarse un poco compleja, si tomamos en cuenta que muchas veces solemos expresarnos de manera ambigua al dar una indicación para realizar una determinada tarea. ya sea, por que utilizamos palabras u expresiones que pueden ser confusas para los demas o bien, por que utilizamos conceptos y palabras desconocidos para algunas personas que no tienen conocimiento en el campo.
\\ Por esta razón, se solicito a los estudiantes que realizaramos esta actividad tratando de ser lo mas claro posible.
\\ A continuación propondré a 3 personas realizar un algoritmo usando una serie de instrucciones con el proposito de que hagan explicitamente lo contenido en el documento, sin que soliciten aclaraciones o ayuda, el algoritmo y los elementos requeridos para el desarrollo de este ejercicio seran explicados mas adelante.

\section{Sección de contenido} \label{contenido}

\subsection{Explicación de la actividad}
La actividad consiste en que la persona que va a realizar el algoritmo tendra 1 hoja de papel y 2 tarjetas de igual tamaño y peso, la posicion inicial de los elementos sera la siguiente:\\
\newline
1. La hoja de papel y las tarjetas estaran ubicadas en una superficie plana y firme.\\
2. La hoja de papel se colocara sobre las tarjetas de modo que las tarjetas esten cubiertas totalmente por la hoja de papel.\\
\newline
El objetivo de esta actividad es que las personas consigan formar un triangulo con ambas tarjetas sobre la hoja de papel siguiendo las instrucciones que les mostrare en el pdf.

\subsection{Algoritmo}


\section{Inclusión de imágenes} \label{imagenes}

En la Figura (\ref{fig:cpplogo}), se presenta el logo de C++ contenido en la carpeta images.

\begin{figure}[h]
\includegraphics[width=4cm]{cpplogo.png}
\centering
\caption{Logo de C++}
\label{fig:cpplogo}
\end{figure}

Las secciones (\ref{intro}), (\ref{contenido}) y (\ref{imagenes}) dependen del estilo del documento.

\bibliographystyle{IEEEtran}
\bibliography{references}

\end{document}
