\documentclass{article}
\usepackage[utf8]{inputenc}
\usepackage[spanish]{babel}
\usepackage{listings}
\usepackage{graphicx}
\graphicspath{ {images/} }
\usepackage{cite}

\begin{document}

\begin{titlepage}
    \begin{center}
        \vspace*{1cm}
            
        \Huge
        \textbf{Informe 1 Informatica II}
            
        \vspace{0.5cm}
        \LARGE
        Algoritmo de la hoja y 2 tarjetas
            
        \vspace{1.5cm}
            
        \textbf{Andrés Felipe Rendón Villada}
            
        \vfill
            
        \vspace{0.8cm}
            
        \Large
        Despartamento de Ingeniería Electrónica y Telecomunicaciones\\
        Universidad de Antioquia\\
        Medellín\\
        Marzo de 2021
            
    \end{center}
\end{titlepage}

\tableofcontents
\newpage
\section{Sección introductoria}\label{intro}
La elaboración de los algoritmos puede tornarse un poco compleja, si tomamos en cuenta que muchas veces solemos expresarnos de manera ambigua al dar una indicación para realizar una determinada tarea. ya sea, por que utilizamos palabras u expresiones que pueden ser confusas para los demas o bien, por que utilizamos conceptos y palabras desconocidos para algunas personas que no tienen conocimiento en el campo.
\\ Por esta razón, se solicito a los estudiantes que realizaramos esta actividad tratando de ser lo mas claro posible.
\\ A continuación propondré a 3 personas realizar un algoritmo usando una serie de instrucciones con el proposito de que hagan explicitamente lo contenido en el documento, sin que soliciten aclaraciones o ayuda, el algoritmo y los elementos requeridos para el desarrollo de este ejercicio seran explicados mas adelante.

\section{Sección de contenido} \label{contenido}

\subsection{Explicación de la actividad}
La actividad consiste en que la persona que va a realizar el algoritmo tendra 1 hoja de papel y 2 tarjetas de igual tamaño y peso, la posicion inicial de los elementos sera la siguiente:\\
\newline
1. La hoja de papel y las tarjetas estaran ubicadas en una superficie plana y firme.\\
2. La hoja de papel se colocara sobre las tarjetas de modo que las tarjetas esten cubiertas totalmente por la hoja de papel.\\
\newline
El objetivo de esta actividad es que las personas consigan formar un triangulo con ambas tarjetas sobre la hoja de papel siguiendo las instrucciones que les mostrare en el pdf.

\subsection{Algoritmo}
Posición inicial: las tarjetas deben de estar debajo de la hoja en una superficie plana, de modo que no se vean las tarjetas.\\
\newline
Únicamente usaremos la mano dominante.\\
\newline
1. Usando su mano dominante, levante la hoja de la superficie y póngala continua o de seguido de las tarjetas. (la hoja ya no debe de cubrir las tarjetas).\\
\newline
2. tome las tarjetas con su mano (seguiremos usando solo la mano dominante).\\
\newline
3. Reubique las tarjetas sobre la hoja, ponga una junto a la otra, de modo que las partes verticales (el lado más largo) tengan contacto-coincidan entre si.\\
\newline
4. Separe ambas tarjetas de manera que el espacio entre ambas sea el grosor de 1 dedo (sin retirar las tarjetas de la hoja).\\
\newline
5.	Tome una tarjeta usando el dedo pulgar, el dedo índice y el dedo medio y ubíquela entre el dedo índice y el dedo medio. Manténgala presionada con el dedo índice y el dedo medio; de tal manera que la tarjeta quede inmóvil.\\
\newline
6.	Con el dedo índice y pulgar tome la 2da tarjeta, pero manteniendo la presión del dedo índice y dedo medio; es decir sin soltar 1er tarjeta.\\
\newline
7.	Apoye firmemente su mano sobre la superficie plana y baje sus dedos por la tarjeta de forma que las tarjetas, la palma de su mano y dedos que queden apoyados sobre la hoja sin soltar las tarjetas.\\
\newline
8.	Incline leve mente las tarjetas de manera que una quede apoyada sobre la otra.\\
\newline
9.	Intente formar un triángulo con ambas tarjetas moviendo sus dedos y levantando un poco su muñeca para conseguirlo.\\
\newline
10.	Una vez el triángulo formado por ambas tarjetas este lo suficiente mente firme retire sus dedos lentamente evitando tirar las tarjetas.\\
\newline
\section{Inclusión de imágenes} \label{imagenes}

En la Figura (\ref{fig:cpplogo}), se presenta el logo de C++ contenido en la carpeta images.

\begin{figure}[h]
\includegraphics[width=4cm]{cpplogo.png}
\centering
\caption{Logo de C++}
\label{fig:cpplogo}
\end{figure}

Las secciones (\ref{intro}), (\ref{contenido}) y (\ref{imagenes}) dependen del estilo del documento.

\bibliographystyle{IEEEtran}
\bibliography{references}

\end{document}
